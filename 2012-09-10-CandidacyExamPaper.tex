\documentclass[10pt,journal,a4paper]{IEEEtran}

\newcommand\MYhyperrefoptions{bookmarks=true,bookmarksnumbered=true,
pdfpagemode={UseOutlines},plainpages=false,pdfpagelabels=true,
colorlinks=true,linkcolor={black},citecolor={black},pagecolor={black},
urlcolor={black},
pdftitle={Metaprogramming with Macros},
pdfsubject={Metaprogramming with Macros},
pdfauthor={Eugene Burmako},
pdfkeywords={metaprogramming, macros, quasiquotes, hygiene, referential transparency}}

\begin{document}

\title{Metaprogramming with Macros}

\author{Eugene Burmako% <-this % stops a space
\\ \hskip90pt LAMP, I\&C,  EPFL % no idea why I need an hskip, but I'm not a latex whiz
\thanks{\normalsize Proposal submitted to committee: September 3rd, 2012; Candidacy exam date: September 10th, 2012; Candidacy exam committee: Christoph Koch, Martin Odersky, Viktor Kuncak.}%
\IEEEcompsocitemizethanks{\IEEEcompsocthanksitem }
\thanks{\large This research plan has been approved:}%
\IEEEcompsocitemizethanks{\IEEEcompsocthanksitem \large}
\IEEEcompsocitemizethanks{\IEEEcompsocthanksitem \large}
\IEEEcompsocitemizethanks{\IEEEcompsocthanksitem \large Date:\hfill------------------------------------}
\IEEEcompsocitemizethanks{\IEEEcompsocthanksitem \large}
\IEEEcompsocitemizethanks{\IEEEcompsocthanksitem \large}
\IEEEcompsocitemizethanks{\IEEEcompsocthanksitem \large Doctoral candidate:\hfill------------------------------------}
\IEEEcompsocitemizethanks{\IEEEcompsocthanksitem \footnotesize \hfill                      (name and signature)\hspace{1.5cm}\hfill}
\IEEEcompsocitemizethanks{\IEEEcompsocthanksitem \large}
\IEEEcompsocitemizethanks{\IEEEcompsocthanksitem \large}
\IEEEcompsocitemizethanks{\IEEEcompsocthanksitem \large Thesis director:\hfill------------------------------------}
\IEEEcompsocitemizethanks{\IEEEcompsocthanksitem \footnotesize \hfill                      (name and signature)\hspace{1.5cm}\hfill}
\IEEEcompsocitemizethanks{\IEEEcompsocthanksitem \large}
\IEEEcompsocitemizethanks{\IEEEcompsocthanksitem \large}
\IEEEcompsocitemizethanks{\IEEEcompsocthanksitem \large Thesis co-director:\hfill------------------------------------}
\IEEEcompsocitemizethanks{\IEEEcompsocthanksitem \footnotesize (if applicable)\hfill  (name and signature)\hspace{1.5cm}\hfill}
\IEEEcompsocitemizethanks{\IEEEcompsocthanksitem \large}
\IEEEcompsocitemizethanks{\IEEEcompsocthanksitem \large}
\IEEEcompsocitemizethanks{\IEEEcompsocthanksitem \large Doct. prog. director:\hfill------------------------------------}
\IEEEcompsocitemizethanks{\IEEEcompsocthanksitem \footnotesize (R. Urbanke)  \hfill                    (signature)\hspace{1.5cm}\hfill}
\IEEEcompsocitemizethanks{\IEEEcompsocthanksitem \large}
\IEEEcompsocitemizethanks{\IEEEcompsocthanksitem \large}
\IEEEcompsocitemizethanks{\IEEEcompsocthanksitem \tiny EDIC-ru/05.05.2009}
}

\markboth{EDIC Research Proposal}%
{Shell \MakeLowercase{\textit{et al.}}: EDIC Research Proposal}

\IEEEcompsoctitleabstractindextext{%
\begin{abstract}
About 60 to 100 words describing what this 4--8 page write-up is about. Start
with an abstract that clearly identifies the problem, the contribution of
the paper, and the results (if any) or ongoing research.
This is not a whole lot of words.  This abstract is
48 words.
\end{abstract}

\begin{IEEEkeywords}
metaprogramming, macros, quasiquotes, hygiene, referential transparency
\end{IEEEkeywords}}

\maketitle
\IEEEdisplaynotcompsoctitleabstractindextext
\IEEEpeerreviewmaketitle

\section{Introduction}

\IEEEPARstart{T}{his} template is not trying to suggest a specific
style for the content of your candidacy exam/thesis proposal paper; it
only concerns the style of the writeup itself.  This template is in LaTeX.
We encourage you to use it for your write-up.

If you insist on using a different word processing program please
match the style as much as possible.  Here is the basic format: The
paper size is A4, the font is 10 point,
single-spacing, ``times roman like.''
Two columns, each 8.75~cm wide, with 1.5~cm left/right margins.
Sections and sub-sections have bold numbered headings.  Figures have
captions underneath, with a bold label and number, and also a short
caption sentence.  Ditto for tables.  The text itself is a mix of standard
paragraphs, bullets, numbered lists, and equations where appropriate.
References are at the end, in a numbered format, in a slightly smaller
font.

The overall length of this write-up should be 4-8 pages.

\subsection{Purpose of this Write-Up}
This write-up serves two purposes. First, it forms the basis for
your candidacy exam. As such you should summarize the three papers selected
by your advisor and yourself, and analyze as well as discuss them critically.
Second, the write-up
is also your thesis proposal.  Therefore, the last one or two pages should
be dedicated to your own preliminary work. A road-map of how you plan to
advance the state of the art in your chosen area should also be given.
For further details please consult the document ``PhD Candidacy Exam Overview.'' You can find the latest
version at {\tt http://phd.epfl.ch/page57746-en.html}.

\subsection{What to Put into the Introduction}
Describe briefly the context, the problem, shortcomings in prior
approaches, and your proposed approach and solution. Forecast results.

\section{Survey of the Selected Papers: The Giants upon whose shoulders I will stand}
Background --- Describe the three papers in detail, the problem they
tackle, the solutions and results, and their shortcomings, and how they
relate to your work.  This part builds the basis for the oral candidacy exam.

\section{Research Proposal: How I Plan to Revolutionize the World}
Describe your own work and include a  summary and references. Write how you
propose to advance the state of the art given the background. What
is new technically? How does it improve over prior work?
Summarize, suggest an approximate timeline, and list references.

\begin{thebibliography}{1}

\bibitem{mittelbach}
F.~Mittelbach, M.~Goossens, J.~Braams, D.~Carlisle, and C.~Rowley,
\emph{The {\LaTeX}} Companion, 2nd~ed.\hskip 1em plus
0.5em minus 0.4em\relax Addison-Wesley Professional, 2004.

\end{thebibliography}

\end{document}
