\documentclass[10pt,journal,a4paper]{IEEEtran}

\usepackage{lipsum}
\usepackage{array}
\usepackage{mdframed}
\newmdenv[linewidth=0.5pt, innerleftmargin=1mm, innerrightmargin=1mm, innertopmargin=1mm, innerbottommargin=1mm, leftmargin=0mm, rightmargin=0mm]{listing}
\usepackage{changepage}

\newcommand\MYhyperrefoptions{bookmarks=true,bookmarksnumbered=true,
pdfpagemode={UseOutlines},plainpages=false,pdfpagelabels=true,
colorlinks=true,linkcolor={black},citecolor={black},pagecolor={black},
urlcolor={black},
pdftitle={Metaprogramming with Macros},
pdfsubject={Metaprogramming with Macros},
pdfauthor={Eugene Burmako},
pdfkeywords={metaprogramming, macros, quasiquotes, hygiene, referential transparency}}

\begin{document}

\title{Metaprogramming with Macros}

\author{Eugene Burmako% <-this % stops a space
\\ \hskip90pt LAMP, I\&C,  EPFL % no idea why I need an hskip, but I'm not a latex whiz
\thanks{\normalsize Proposal submitted to committee: September 3rd, 2012; Candidacy exam date: September 10th, 2012; Candidacy exam committee: Christoph Koch, Martin Odersky, Viktor Kuncak.}%
\IEEEcompsocitemizethanks{\IEEEcompsocthanksitem }
\thanks{\large This research plan has been approved:}%
\IEEEcompsocitemizethanks{\IEEEcompsocthanksitem \large}
\IEEEcompsocitemizethanks{\IEEEcompsocthanksitem \large}
\IEEEcompsocitemizethanks{\IEEEcompsocthanksitem \large Date:\hfill------------------------------------}
\IEEEcompsocitemizethanks{\IEEEcompsocthanksitem \large}
\IEEEcompsocitemizethanks{\IEEEcompsocthanksitem \large}
\IEEEcompsocitemizethanks{\IEEEcompsocthanksitem \large Doctoral candidate:\hfill------------------------------------}
\IEEEcompsocitemizethanks{\IEEEcompsocthanksitem \footnotesize \hfill                      (name and signature)\hspace{1.5cm}\hfill}
\IEEEcompsocitemizethanks{\IEEEcompsocthanksitem \large}
\IEEEcompsocitemizethanks{\IEEEcompsocthanksitem \large}
\IEEEcompsocitemizethanks{\IEEEcompsocthanksitem \large Thesis director:\hfill------------------------------------}
\IEEEcompsocitemizethanks{\IEEEcompsocthanksitem \footnotesize \hfill                      (name and signature)\hspace{1.5cm}\hfill}
\IEEEcompsocitemizethanks{\IEEEcompsocthanksitem \large}
\IEEEcompsocitemizethanks{\IEEEcompsocthanksitem \large}
\IEEEcompsocitemizethanks{\IEEEcompsocthanksitem \large Thesis co-director:\hfill------------------------------------}
\IEEEcompsocitemizethanks{\IEEEcompsocthanksitem \footnotesize (if applicable)\hfill  (name and signature)\hspace{1.5cm}\hfill}
\IEEEcompsocitemizethanks{\IEEEcompsocthanksitem \large}
\IEEEcompsocitemizethanks{\IEEEcompsocthanksitem \large}
\IEEEcompsocitemizethanks{\IEEEcompsocthanksitem \large Doct. prog. director:\hfill------------------------------------}
\IEEEcompsocitemizethanks{\IEEEcompsocthanksitem \footnotesize (R. Urbanke)  \hfill                    (signature)\hspace{1.5cm}\hfill}
\IEEEcompsocitemizethanks{\IEEEcompsocthanksitem \large}
\IEEEcompsocitemizethanks{\IEEEcompsocthanksitem \large}
\IEEEcompsocitemizethanks{\IEEEcompsocthanksitem \tiny EDIC-ru/05.05.2009}
}

\markboth{EDIC Research Proposal}%
{Shell \MakeLowercase{\textit{et al.}}: EDIC Research Proposal}

\IEEEcompsoctitleabstractindextext{%
\begin{abstract}
Macros realize the notion of \emph{textual abstraction}.
Textual abstraction consists of recognizing pieces of text
that match a specification and replacing them according
to a procedure.

In the focus of the study are syntactic macros in lexically
scoped programming languages. We identify the problems
of \emph{hygiene} and \emph{referential transparency}
and describe the solutions employed in Template Haskell \cite{sheard02},
Nemerle \cite{skalski04} and Racket \cite{barzilay11}.

We discuss integration of hygienic macros into statically typed languages
and propose to improve upon state of the art by providing
a type system for syntax templates and uncovering synergies with
high-level language features such as path-dependent types and implicits \cite{odersky10}.
\end{abstract}

\begin{IEEEkeywords}
metaprogramming, macros, quasiquotes, hygiene, referential transparency
\end{IEEEkeywords}}

\maketitle
\IEEEdisplaynotcompsoctitleabstractindextext
\IEEEpeerreviewmaketitle

\section{Introduction}

% \subsection{Procedural abstraction}

\IEEEPARstart{P}{rocedural} abstraction is pervasive.
Factoring out parameterized fragments of programs into procedures
is a conventional best practice.

Modern programming languages integrate the notion of procedures into their semantics.
Procedures are viewed as independent programs that can communicate with the main program.
As of such they can be manipulated as units, and big procedures can be built from the smaller ones.
This is a powerful way to manage complexity of software systems.

However procedural abstraction is sometimes not enough, because
its manifestations are bound by language syntax and it operates within the semantics
of the language.

For example, in most programming languages it is impossible to define short-circuiting logical operators
as procedures, because procedures are usually not in control of operational semantics.
Another example in this vein is a C-like \texttt{for} loop, which supports
optional prologue that introduces variables visible in its body. Procedures typically cannot abstract
over variable bindings, so they cannot express this language construct.

% \subsection{Textual abstraction}

\emph{Textual abstraction} consists of recognizing pieces of text
that match a specification and replacing them according
to a procedure.
Matched fragments are referred to as macro calls or macro applications, and
procedures that transform them are dubbed macros or macro transformers.
The process of applying macros is called macro expansion \cite{kohlbecker86}.

Having means of textual abstraction in their toolbox, programmers can use a multitude of techniques,
some of which are:
\begin{itemize}
\item reification (providing programs with ways to treat code as data),
\item language virtualization (overloading/overriding semantics of the original programming language
to enable deep embedding of DSLs),
\item programmable optimization (application of optimizations such as inlining
or fusion based on knowledge about the program being optimized),
\item static verification (using reified representation of the program
and, possibly, its contracts defined alongside the program, to verify invariants
at compile time),
\item algorithmic program construction (generation of code that is tedious to write with
the abstractions supported by a programming language).
\end{itemize}

One possibility to implement a macro system is having it as a standalone tool
operating on character streams. This gives rise to lexical macros.
Such a design warrants simplicity of the implementation,
but undermines robustness, because macros operating on lexical level
have no mechanism that prevents generation of syntactically invalid programs.

Another approach would be to integrate a macro expander into the compiler and
have macros work with syntax trees, introducing \emph{syntactic macros}.
In this model macro application is a node in the program tree,
and macro expansion produces a new node that replaces the macro application
without distorting the structure of the program.

Problems inherent to syntactic macros can be divided into two categories:
inadvertent variable capture and expansion into semantically invalid code.
The rest of the paper dwells upon these challenges.

\begin{figure*}[t]
\begin{listing}
\normalsize

\begin{tabular}{p{4.0cm} p{15cm}}\\
 &
\begin{verbatim}
(let ((x 40) (y 2)) (print (+ x y)))

((lambda (x y) (print (+ x y))) (40 2))
\end{verbatim}
\end{tabular}

\begin{center}
a) Examples of a macro application and a macro expansion of the \texttt{let} macro
\end{center}

\begin{tabular}{p{8.5cm} p{8.5cm}}\\
\begin{verbatim}
(defmacro let args
  (cons
    (cons 'lambda
          (cons (map car (car args))
                (cdr args)))
    (map cadr (car args))))
\end{verbatim}
&
\begin{verbatim}
(defmacro let (decls body)
 `(
   (lambda
     ,(map car decls)   ;; (x y)
     ,body)             ;; (print (+ x y))
   ,@(map cadr decls))) ;; (40 2)
\end{verbatim}\\
b) Creates the result at low level (manipulates S-expressions
with standard symbol and list processing functions)
&
c) Uses quasiquotes \cite{bawden99} as a templating mechanism
(backquote introduces a static code template, commas splice dynamic values
into the template)
\end{tabular}

\begin{tabular}{p{3.5cm} p{13.5cm}}\\
 &
\begin{verbatim}
(define-syntax let
  (syntax-rules ()
    ((let ((name expr) ...) body ...)
    ((lambda (name ...) body ...) expr ...))))
\end{verbatim}
\end{tabular}

\begin{adjustwidth}{4cm}{0pt}
d) Macro-by-example notation \cite{kohlbecker87} (uses a tree matching macro that\\
can extract and reassemble fragments of syntax objects; ellipses cap-\\ture
recurrent parts of the input)
\end{adjustwidth}

\begin{tabular}{p{2.5cm} p{14.5cm}}\\
 &
\begin{verbatim}
(define-syntax (let stx)
  (syntax-parse stx
    ((let ((name:identifier expr:expr) ...) body:expr ...)
     #:fail-when (check-duplicate #'(var ...))
                  "duplicate variable name"
     #'((lambda (name ...) body ...) expr ...))))
\end{verbatim}
\end{tabular}

\begin{adjustwidth}{3.2cm}{0pt}
e) Macro-by-example notation augmented with a syntax specification \cite{culpepper10}
(colons \\denote syntax classes, which are first-class and can be built from the ground up).
\end{adjustwidth}
\end{listing}
\end{figure*}

\begin{figure*}
\hskip3.95cm
\normalsize Figure 1. Assorted implementations of the \texttt{let} macro in Lisp dialects
% saves an empty line
% \begin{center}
% \normalsize Figure 1. Assorted implementations of the \texttt{let} macro in Lisp dialects
% \end{center}
\end{figure*}

\begin{figure*}[t]
\begin{listing}
\normalsize

\begin{tabular}{p{5.4cm} p{15cm}}\\
 &
\begin{verbatim}
(defmacro or (x y)
  '(let ((temp ,x))
    (if temp temp ,y)))

(or 42 "the result is 42")
\end{verbatim}
\end{tabular}

\begin{center}
a) A simple yet erroneous implementation of the \texttt{or} macro
\end{center}

\begin{tabular}{p{8.5cm} p{8.5cm}}\\
\begin{verbatim}
(let ((temp "451 Fahrenheit"))
  (or null temp))

(let ((temp "451 Fahrenheit"))
  (let ((temp null))
    (if temp temp temp)))
\end{verbatim}
&
\begin{verbatim}
(let ((if hijacked))
  (or true false))

(let ((if hijacked))
  (let ((temp true))
    (if temp temp false)))
\end{verbatim}\\
b) Violation of hygiene \cite{kohlbecker86}: binding established during expansion
affects call site
&
c) Violation of referential transparency \cite{dybvig92}: binding established during expansion
is affected by call site
\end{tabular}

\begin{tabular}{p{2.8cm} p{15cm}}\\
 &
\begin{verbatim}
(define-syntax forever
  (syntax-rules ()
    ((forever body ...)
     (call/cc (lambda (abort)
                (let loop () body ... (loop)))))))

(forever (print 4) (print 2) (abort))
\end{verbatim}
\end{tabular}

\begin{adjustwidth}{2.5cm}{0pt}
d) Intentional variable capture: infinite looping construct \texttt{forever} provides
a predefi-\\ned identifier to break from the loop. \texttt{abort} is introduced during
macro expansion,\\ but it should be visible to the body of the loop, which
comes from the macro call site.
\end{adjustwidth}

\end{listing}
\end{figure*}

\begin{figure*}
\hskip5.05cm
\normalsize Figure 2. Intentional and unintentional variable capture
% saves an empty line
% \begin{center}
% \normalsize Figure 2. Intentional and unintentional variable capture
% \end{center}
\end{figure*}

\section{Examples}

\texttt{let} is a language construct typical to functional programming. It introduces
a scope for a computation and brings temporary variables with provided values into that scope.
To implement \texttt{let} the compiler might wrap the computation in a lambda abstraction
and apply it right away (Figure 1a).

This notion cannot be abstracted procedurally, because the body of the computation typically
contains free variables. However textual abstraction fits the bill, because macros can manipulate
the program on a level where bindings don't exist and therefore don't impose restrictions.
To set up a stage for further chapters, let's implement the \texttt{let} macro in Lisp.

The most straightforward solution to the problem is a low-level macro transformer (Figure 1b).
It takes an S-expression that represents a macro application, destructures it using standard
list manipulation functions, such as \texttt{car} and \texttt{cdr}, and creates a new S-expression
with \texttt{cons}. Even in this simple example this notation is very noisy. It's quite difficult
to figure out expected shapes of input and output expressions from the imperative algorithm.

Quasiquotes \cite{bawden99} make it possible to reduce obscurity of the macro by providing
a domain-specific language for syntax templates (Figure 1c). The quasiquote operator (\texttt{`})
demarcates a static template. Quasiquoted code is inserted verbatim into the output
(that's why there's no longer need in explicit \texttt{cons}'ing). Unquote operators (\texttt{,}
and \texttt{,@}) interrupt a quasiquote, producing "holes" filled in with dynamically calculated
data. For example, for \texttt{let} we statically know the shape of code to produce (an application
of a lambda abstraction) - this makes up the static part of the quasiquote. On the other hand,
body and parameters of the lambda as well as the arguments of the application may vary from
expansion to expansion - this is the dynamic part.

Another simplification of the macro can be achieved with MBE, the macro-by-example notation
\cite{kohlbecker87}. In their seminal work Kohlbecker and Wand came up with a specification
of a pattern matcher that matches singular and repetitive parts of S-expressions.
Identifiers which appear in the input pattern are treated as pattern variables,
ellipses (\texttt{...}) used as a last element of a list that contains pattern variables
denote repetition and, when nested, can capture lists of arbitrary depth.
The revised version of the \texttt{let} macro is particularly minimalistic (Figure 1d).

A recent development of the MBE syntax has been proposed by Culpepper and Felleisen \cite{culpepper10}.
Their refinement addresses the need for principled input validation and error reporting. Indeed,
MBE covers the success path, but doesn't help with detecting errors. For example,
duplicate identifier names as in \texttt{(let ((x 40) (x 2)) (print (+ x y))} will go
unnoticed until the compiler gets to the resulting lambda form, which will produce
confusing error messages. Authors enhance MBE with both declarative and procedural
means of validation (Figure 1e). Colons next to the names of pattern variables denote syntax classes,
which put restrictions on the shape of the variables, \texttt{\#:fail-when} clauses can contain
validation code and error messages (this doesn't cover all the capabilities of syntax specifications,
please, refer to \cite{culpepper10} for details). These validation facilities can be packed
into custom syntax classes, which can be built from the ground up.

\section{Bindings}

Syntactic macros operate on ASTs, so they cannot produce syntactically invalid code,
but nothing prevents such macros from making semantic errors, i.e. expanding into
syntactically valid nonsense.
The most blatant mistakes will result in type errors, however there's an entire class
of oversights that might silently change the behavior of the program.

When writing procedures in lexically scoped languages, programmers typically don't need to think
about name clashes between variables declared inside procedures and at their call sites.
Except for recursion, scopes of procedure bodies and procedure call sites are different, so
neither variables defined inside the procedures can change the semantics at the call site, nor vice versa.

In a macro-enabled language, especially in presense of quasiquotes which make
macros look like normal procedures, this intuition becomes compromised,
because after macro expansion the scopes of macro definition and macro call site
are mechanically merged by the expander.

Conider the \texttt{or} macro shown on Figure 2a. This macro picks one of its two arguments
based on the truthiness of the first argument. To prevent double evaluation, the macro introduces
a temporary variable \texttt{temp} that holds the result of evaluation of the first argument.
The temporary variable is then tested with \texttt{if}, which selects the resulting value.

As simple as this code can be, it is also incorrect, having two potential bugs.

The first problem happens when the call site defines its own variable named \texttt{temp}
and relies on it in the second argument of a call to \texttt{or} (Figure 2b). After the expansion, two
\texttt{temp}s clash, which produces incorrect results if the first argument of the call is falsy.
This problem is dubbed \emph{hygiene violation}, and the macro is said to introduce the temporary
variable \emph{unhygienically}.

In his thesis \cite{kohlbecker86} Kohlbecker defines the hygiene condition
and presents a macro system that automatically prevents such naming collisions:
\begin{quote}
Hygiene Condition for Macro Expansion.

Generated identifiers that become binding instances in the
completely expanded program must bind only identifiers that are
generated during the same transcription step.
\end{quote}

The second problem is in a sense dual to the first one. It happens when the call site
redefines one of the procedures or macros used in the expansion. For example, on Figure 2c
the call site hijacks the meaning of \texttt{if}, which destroys the original intent of the
macro author. This is a \emph{referential opaqueness} problem.

Dybvig et al. \cite{dybvig92} build on Kohlbecker's work and devise a macro expander that
automatically avoids this class of errors:
\begin{quote}
Macros defined in [our] high-level specification language are
referentially transparent in the sense that a macro-introduced
identifier refers to the binding lexically visible where the macro
definition appears rather than to the top-level binding or to the
binding visible where the macro call appears.
\end{quote}

This notion is also called \emph{cross-stage persistence} in a sense that bindings in place
at the compilation stage are persisted into the runtime stage. When viewed in this light,
it becomes apparent that referential transparency for macros can only work for references
to top-level definitions. Therefore common practice is to report cross-stage references to local variables
as errors \cite{dybvig92}.

Despite the conveniency of automatic segregation of the scopes of macro definitions and macro call sites,
at times it is necessary to have them melded.
A looping macro \texttt{forever} from Figure 2d demonstrates the need for this. Expansion of this macro
is an infinite loop that provides a magic identifier \texttt{abort} to exit the loop.
In this case automatic hygiene facility will do harm, making the body of the loop unable to see
the loop control lever.

Because of similar situations some programmers argue in favor of manual control over
the scoping discipline, proposing manual (or macro-powered!) renaming of identifiers whose capture would be
undesired \cite{clinger91,hoyte08}.

Another popular line of thought favors further empowerment of macro systems
to minimize the necessity for low-level tinkering.
In fact, one of the macro systems reviewed below provides a design pattern \cite{barzilay11}
that solves the \texttt{abort} challenge in a fully hygienic way (i.e. without introducing
identifiers with manually provided names).

\begin{thebibliography}{99}

\bibitem{sheard02}
T.~Sheard and S.~Peyton Jones,
Template meta-programming for Haskell.
ACM SIGPLAN Notices, 2002.

\bibitem{skalski04}
K.~Skalski, M.~Moskal and P.~Olszta,
Meta-programming in Nemerle.
Generative Programming and Component Engineering, 2004.

\bibitem{barzilay11}
E.~Barzilay, R.~Culpepper and M.~Flatt,
Keeping it Clean with Syntax Parameters.
Scheme and Functional Programming Workshop, 2011.

\bibitem{odersky10}
M.~Odersky, L.~Spoon and B.~Venners,
Programming in Scala 2nd Edition.
Artima, 2010.

\bibitem{kohlbecker86}
E.~Kohlbecker,
Syntactic Extensions in the Programming Language Lisp.
PhD thesis, Indiana University, 1986.

\bibitem{dybvig92}
R.~Dybvig, R.~Hieb and  C.~Bruggeman,
Syntactic Abstraction in Scheme.
Lisp and Symbolic Computations, 1992.

\bibitem{bawden99}
A.~Bawden,
Quasiquotation in Lisp,
Partial Evaluation and SemanticBased Program Manipulation, 1999.

\bibitem{kohlbecker87}
E.~Kohlbecker, M.~Wand,
Macro-by-Example: Deriving Syntactic Transformations from their Specifications,
Principles of Programming Languages, 1987.

\bibitem{culpepper10}
R.~Culpepper, M.~Felleisen,
Fortifying Macros,
International Conference on Functional Programming, 2010.

\bibitem{hoyte08}
D.~Hoyte,
Let Over Lambda,
Lulu, 2008.

\bibitem{clinger91}
W.~Clinger,
Hygienic macros through explicit renaming,
ACM SIGPLAN Lisp Pointers, 1991.

\end{thebibliography}

\end{document}
