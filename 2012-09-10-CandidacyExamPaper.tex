\documentclass[10pt,journal,a4paper]{IEEEtran}

\newcommand\MYhyperrefoptions{bookmarks=true,bookmarksnumbered=true,
pdfpagemode={UseOutlines},plainpages=false,pdfpagelabels=true,
colorlinks=true,linkcolor={black},citecolor={black},pagecolor={black},
urlcolor={black},
pdftitle={Metaprogramming with Macros},
pdfsubject={Metaprogramming with Macros},
pdfauthor={Eugene Burmako},
pdfkeywords={metaprogramming, macros, quasiquotes, hygiene, referential transparency}}

\begin{document}

\title{Metaprogramming with Macros}

\author{Eugene Burmako% <-this % stops a space
\\ \hskip90pt LAMP, I\&C,  EPFL % no idea why I need an hskip, but I'm not a latex whiz
\thanks{\normalsize Proposal submitted to committee: September 3rd, 2012; Candidacy exam date: September 10th, 2012; Candidacy exam committee: Christoph Koch, Martin Odersky, Viktor Kuncak.}%
\IEEEcompsocitemizethanks{\IEEEcompsocthanksitem }
\thanks{\large This research plan has been approved:}%
\IEEEcompsocitemizethanks{\IEEEcompsocthanksitem \large}
\IEEEcompsocitemizethanks{\IEEEcompsocthanksitem \large}
\IEEEcompsocitemizethanks{\IEEEcompsocthanksitem \large Date:\hfill------------------------------------}
\IEEEcompsocitemizethanks{\IEEEcompsocthanksitem \large}
\IEEEcompsocitemizethanks{\IEEEcompsocthanksitem \large}
\IEEEcompsocitemizethanks{\IEEEcompsocthanksitem \large Doctoral candidate:\hfill------------------------------------}
\IEEEcompsocitemizethanks{\IEEEcompsocthanksitem \footnotesize \hfill                      (name and signature)\hspace{1.5cm}\hfill}
\IEEEcompsocitemizethanks{\IEEEcompsocthanksitem \large}
\IEEEcompsocitemizethanks{\IEEEcompsocthanksitem \large}
\IEEEcompsocitemizethanks{\IEEEcompsocthanksitem \large Thesis director:\hfill------------------------------------}
\IEEEcompsocitemizethanks{\IEEEcompsocthanksitem \footnotesize \hfill                      (name and signature)\hspace{1.5cm}\hfill}
\IEEEcompsocitemizethanks{\IEEEcompsocthanksitem \large}
\IEEEcompsocitemizethanks{\IEEEcompsocthanksitem \large}
\IEEEcompsocitemizethanks{\IEEEcompsocthanksitem \large Thesis co-director:\hfill------------------------------------}
\IEEEcompsocitemizethanks{\IEEEcompsocthanksitem \footnotesize (if applicable)\hfill  (name and signature)\hspace{1.5cm}\hfill}
\IEEEcompsocitemizethanks{\IEEEcompsocthanksitem \large}
\IEEEcompsocitemizethanks{\IEEEcompsocthanksitem \large}
\IEEEcompsocitemizethanks{\IEEEcompsocthanksitem \large Doct. prog. director:\hfill------------------------------------}
\IEEEcompsocitemizethanks{\IEEEcompsocthanksitem \footnotesize (R. Urbanke)  \hfill                    (signature)\hspace{1.5cm}\hfill}
\IEEEcompsocitemizethanks{\IEEEcompsocthanksitem \large}
\IEEEcompsocitemizethanks{\IEEEcompsocthanksitem \large}
\IEEEcompsocitemizethanks{\IEEEcompsocthanksitem \tiny EDIC-ru/05.05.2009}
}

\markboth{EDIC Research Proposal}%
{Shell \MakeLowercase{\textit{et al.}}: EDIC Research Proposal}

\IEEEcompsoctitleabstractindextext{%
\begin{abstract}
Macros realize the notion of textual abstraction.
Textual abstraction consists of recognizing pieces of text
that match a specification and replacing them according
to a procedure.

In the focus of the study are syntactic macros in lexically
scoped programming languages. We identify the problems
of \emph{hygiene} and \emph{referential transparency}
and describe the solutions employed in Template Haskell \cite{sheard02},
Nemerle \cite{skalski04} and Racket \cite{barzilay11}.

We discuss integration of hygienic macros into statically typed languages
and propose to improve upon state of the art by providing
a type system for syntax templates and uncovering synergies with
high-level language features such as path-dependent types and implicits \cite{odersky10}.
\end{abstract}

\begin{IEEEkeywords}
metaprogramming, macros, quasiquotes, hygiene, referential transparency
\end{IEEEkeywords}}

\maketitle
\IEEEdisplaynotcompsoctitleabstractindextext
\IEEEpeerreviewmaketitle

\section{Introduction}

% \subsection{Procedural abstraction}

\IEEEPARstart{P}{rocedural} abstraction is pervasive.
Factoring out parameterized fragments of programs into procedures
is a conventional best practice.

Modern programming languages integrate the notion of procedures into their semantics.
Procedures are viewed as independent programs that can communicate with the main program.
As of such they can be manipulated as units, and big procedures can be built from the smaller ones.
This is a powerful way to manage complexity of software systems.

However procedural abstraction is sometimes not enough, because
its manifestations are bound by language syntax and it operates within the semantics
of the language.

For example, in most programming languages it is impossible to define short-circuiting logical operators
as procedures, because procedures are usually not in control of operational semantics.
Another example in this vein is a C-like \texttt{for} loop, which supports
optional prologue that introduces variables visible in its body. Procedures typically cannot abstract
over variable bindings, so they cannot express this language construct.

% \subsection{Textual abstraction}

\emph{Textual abstraction} consists of recognizing pieces of text
that match a specification and replacing them according
to a procedure.
Matched fragments are referred to as macro calls or macro applications, and
procedures that transform them are dubbed macros or macro transformers.
The process of applying macros is called macro expansion \cite{kohlbecker86}.

Having means of textual abstraction in their toolbox, programmers can use a multitude of techniques,
some of which are:
\begin{itemize}
\item reification (providing programs with ways to treat code as data),
\item language virtualization (overloading/overriding semantics of the original programming language
to enable deep embedding of DSLs),
\item programmable optimization (application of optimizations such as inlining
or fusion based on knowledge about the program being optimized),
\item static verification (using reified representation of the program
and, possibly, its contracts defined alongside the program, to verify invariants
at compile time),
\item algorithmic program construction (generation of code that is tedious to write with
the abstractions supported by a programming language).
\end{itemize}

One possibility to implement a macro system is having it as a standalone tool
operating on character streams. This gives rise to lexical macros.
Such a design warrants simplicity of the implementation,
but undermines robustness, because macros operating on lexical level
have no mechanism that prevents generation of syntactically invalid programs.

Another approach would be to integrate a macro expander into the compiler and
have macros work with syntax trees, introducing \emph{syntactic macros}.
In this model macro application is a node in the program tree,
and macro expansion produces a new node that replaces the macro application
without distorting the structure of the program.

Problems inherent to syntactic macros can be divided into two categories:
inadvertent variable capture and semantically invalid expansions.
The rest of the papers dwells upon these challenges.

\begin{thebibliography}{1}

\bibitem{sheard02}
T.~Sheard and S.~Peyton Jones,
Template meta-programming for Haskell.
ACM SIGPLAN Notices, 2002.

\bibitem{skalski04}
K.~Skalski, M.~Moskal and P.~Olszta,
Meta-programming in Nemerle.
Third International Conference on
Generative Programming and Component Engineering, 2004.

\bibitem{barzilay11}
E.~Barzilay, R.~Culpepper and M.~Flatt,
Keeping it Clean with Syntax Parameters.
Scheme and Functional Programming Workshop, 2011.

\bibitem{odersky10}
M.~Odersky, L.~Spoon and B.~Venners,
Programming in Scala 2nd Edition.
Artima, 2010.

\bibitem{kohlbecker86}
E.~Kohlbecker,
Syntactic Extensions in the Programming Language Lisp.
PhD thesis, Indiana University, 1986.

\end{thebibliography}

\end{document}
