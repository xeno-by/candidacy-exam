\documentclass[10pt,journal,a4paper]{IEEEtran}

\usepackage{lipsum}
\usepackage{array}
\usepackage{mdframed}
\newmdenv[linewidth=0.5pt, innerleftmargin=1mm, innerrightmargin=1mm, innertopmargin=1mm, innerbottommargin=1mm, leftmargin=0mm, rightmargin=0mm]{listing}
\usepackage{changepage}
\usepackage[noadjust, sort, compress]{cite}
\renewcommand\citeform[1]{#1}
\renewcommand\citeleft{[}
\renewcommand\citeright{]}
\renewcommand\citepunct{,}
\renewcommand\citedash{--}

\newcommand\MYhyperrefoptions{bookmarks=true,bookmarksnumbered=true,
pdfpagemode={UseOutlines},plainpages=false,pdfpagelabels=true,
colorlinks=true,linkcolor={black},citecolor={black},pagecolor={black},
urlcolor={black},
pdftitle={Metaprogramming with Macros},
pdfsubject={Metaprogramming with Macros},
pdfauthor={Eugene Burmako},
pdfkeywords={metaprogramming, macros, quasiquotes, hygiene, referential transparency}}

\begin{document}

\title{Metaprogramming with Macros}

\author{Eugene Burmako% <-this % stops a space
\\ \hskip90pt LAMP, I\&C,  EPFL % no idea why I need an hskip, but I'm not a latex whiz
\thanks{\normalsize Proposal submitted to committee: September 3rd, 2012; Candidacy exam date: September 10th, 2012; Candidacy exam committee: Christoph Koch, Martin Odersky, Viktor Kuncak.}%
\IEEEcompsocitemizethanks{\IEEEcompsocthanksitem }
\thanks{\large This research plan has been approved:}%
\IEEEcompsocitemizethanks{\IEEEcompsocthanksitem \large}
\IEEEcompsocitemizethanks{\IEEEcompsocthanksitem \large}
\IEEEcompsocitemizethanks{\IEEEcompsocthanksitem \large Date:\hfill------------------------------------}
\IEEEcompsocitemizethanks{\IEEEcompsocthanksitem \large}
\IEEEcompsocitemizethanks{\IEEEcompsocthanksitem \large}
\IEEEcompsocitemizethanks{\IEEEcompsocthanksitem \large Doctoral candidate:\hfill------------------------------------}
\IEEEcompsocitemizethanks{\IEEEcompsocthanksitem \footnotesize \hfill                      (name and signature)\hspace{1.5cm}\hfill}
\IEEEcompsocitemizethanks{\IEEEcompsocthanksitem \large}
\IEEEcompsocitemizethanks{\IEEEcompsocthanksitem \large}
\IEEEcompsocitemizethanks{\IEEEcompsocthanksitem \large Thesis director:\hfill------------------------------------}
\IEEEcompsocitemizethanks{\IEEEcompsocthanksitem \footnotesize \hfill                      (name and signature)\hspace{1.5cm}\hfill}
\IEEEcompsocitemizethanks{\IEEEcompsocthanksitem \large}
\IEEEcompsocitemizethanks{\IEEEcompsocthanksitem \large}
\IEEEcompsocitemizethanks{\IEEEcompsocthanksitem \large Thesis co-director:\hfill------------------------------------}
\IEEEcompsocitemizethanks{\IEEEcompsocthanksitem \footnotesize (if applicable)\hfill  (name and signature)\hspace{1.5cm}\hfill}
\IEEEcompsocitemizethanks{\IEEEcompsocthanksitem \large}
\IEEEcompsocitemizethanks{\IEEEcompsocthanksitem \large}
\IEEEcompsocitemizethanks{\IEEEcompsocthanksitem \large Doct. prog. director:\hfill------------------------------------}
\IEEEcompsocitemizethanks{\IEEEcompsocthanksitem \footnotesize (R. Urbanke)  \hfill                    (signature)\hspace{1.5cm}\hfill}
\IEEEcompsocitemizethanks{\IEEEcompsocthanksitem \large}
\IEEEcompsocitemizethanks{\IEEEcompsocthanksitem \large}
\IEEEcompsocitemizethanks{\IEEEcompsocthanksitem \tiny EDIC-ru/05.05.2009}
}

\markboth{EDIC Research Proposal}%
{Shell \MakeLowercase{\textit{et al.}}: EDIC Research Proposal}

\IEEEcompsoctitleabstractindextext{%
\begin{abstract}
Macros realize the notion of \emph{textual abstraction}.
Textual abstraction consists of recognizing pieces of text
that match a specification and replacing them according
to a procedure.

In the focus of the study is semantics of syntactic macros in lexically
scoped programming languages. We highlight the problems
of \emph{hygiene} and \emph{referential transparency}
and describe the solutions employed Template Haskell \cite{sheard02},
Nemerle \cite{skalski04,skalski05} and Racket \cite{barzilay11,flatt10}.

We discuss integration of hygienic macros into statically typed languages
and plan to improve upon state of the art by providing
a flexible type system for syntax templates and uncovering synergies with
high-level language features such as path-dependent types and implicits \cite{odersky10}.
\end{abstract}

\begin{IEEEkeywords}
metaprogramming, macros, quasiquotes, hygiene, referential transparency
\end{IEEEkeywords}}

\maketitle
\IEEEdisplaynotcompsoctitleabstractindextext
\IEEEpeerreviewmaketitle

\section{Introduction}

% \subsection{Procedural abstraction}

\IEEEPARstart{P}{rocedural} abstraction is pervasive.
Factoring out parameterized fragments of programs into procedures
is a conventional best practice.

Modern programming languages integrate the notion of procedures into their semantics.
Procedures are viewed as independent programs that can communicate with the main program.
As of such they can be manipulated as units, and big procedures can be built from the smaller ones.
This is a powerful way to manage complexity of software systems.

However procedural abstraction is sometimes not enough, because
its manifestations are bound by language syntax and it operates within the semantics
of the language.

For example, in most programming languages it is impossible to define short-circuiting logical operators
as procedures, because procedures are usually not in control of operational semantics.
Another example in this vein is a C-like \small \texttt{for} \normalsize loop, which supports
optional prologue that introduces variables visible in its body. Procedures typically cannot abstract
over variable bindings, so they cannot express this language construct.

% \subsection{Textual abstraction}

\emph{Textual abstraction} consists of recognizing pieces of text
that match a specification and replacing them according
to a procedure.
Matched fragments are called macro calls or macro applications, and
procedures that transform them are dubbed macros or macro transformers \cite{kohlbecker86}
(and in general sense these procedures can be referred to as meta-programs \cite{sheard01}).
The process of applying macros is called macro expansion.

Having means of textual abstraction in their toolbox, programmers can use a multitude of techniques,
some of which are:
\begin{itemize}
\item reification (providing code as data),
\item language virtualization (overloading/overriding semantics
to enable deep embedding of DSLs),
\item domain-specific optimization (application of optimizations such as inlining
or fusion based on knowledge about the program being optimized),
\item static verification (using reified representation of the program
and, possibly, its contracts defined alongside the program, to verify invariants
at compile time),
\item algorithmic program construction (generation of code that is tedious to write with
supported abstractions).
\end{itemize}

One possibility to implement a macro system is having it as a standalone tool
operating on character streams. This gives rise to lexical macros.
Such a design warrants simplicity of the implementation,
but undermines robustness, because macros operating on lexical level
have no mechanism that prevents generation of syntactically invalid programs.

Alternative approach is to integrate a macro expander into the compiler and
have macros work with syntax trees, introducing \emph{syntactic macros}.
In this model macro application is a node in the program tree,
and macro expansion produces a new node that replaces the macro application
without distorting the structure of the program.

A significant body of recent research in textual abstraction has been done on syntax
extensibility (\cite{atkinson11, allen09, erdweg11} to name only a few).
This paper however focuses on semantics of macro-enabled languages,
i.e. on preventing macros from expanding into nonsense
and finding synergies with existing language features.

\begin{figure*}[t]
\begin{listing}
\normalsize

\begin{tabular}{p{4.0cm} p{15cm}}\\
 &
\begin{verbatim}
(let ((x 40) (y 2)) (print (+ x y)))

((lambda (x y) (print (+ x y))) (40 2))
\end{verbatim}
\end{tabular}

\begin{center}
a) Examples of a macro application and a macro expansion of the \small \texttt{let} \normalsize macro
\end{center}

\begin{tabular}{p{8.5cm} p{8.5cm}}\\
\begin{verbatim}
(defmacro let args
  (cons
    (cons 'lambda
          (cons (map car (car args))
                (cdr args)))
    (map cadr (car args))))
\end{verbatim}
&
\begin{verbatim}
(defmacro let (decls body)
 `(
   (lambda
     ,(map car decls)   ;; (x y)
     ,body)             ;; (print (+ x y))
   ,@(map cadr decls))) ;; (40 2)
\end{verbatim}\\
b) Creates the result at low level (manipulates S-expressions
with standard symbol and list processing functions)
&
c) Uses quasiquotes \cite{bawden99} as a templating mechanism
(backquote introduces a static code template, commas (also known as
unquotes) insert dynamic values into the template)
\end{tabular}

\begin{tabular}{p{3.5cm} p{13.5cm}}\\
 &
\begin{verbatim}
(define-syntax let
  (syntax-rules ()
    ((let ((name expr) ...) body ...)
    ((lambda (name ...) body ...) expr ...))))
\end{verbatim}
\end{tabular}

\begin{adjustwidth}{4cm}{0pt}
d) Macro-by-example notation \cite{kohlbecker87} (uses a tree matching macro that\\
can extract and reassemble fragments of syntax objects; ellipses cap-\\ture
recurrent parts of the input)
\end{adjustwidth}

\begin{tabular}{p{2.5cm} p{14.5cm}}\\
 &
\begin{verbatim}
(define-syntax (let stx)
  (syntax-parse stx
    ((let ((name:identifier expr:expr) ...) body:expr ...)
     #:fail-when (check-duplicate #'(var ...))
                  "duplicate variable name"
     #'((lambda (name ...) body ...) expr ...))))
\end{verbatim}
\end{tabular}

\begin{adjustwidth}{3.2cm}{0pt}
e) Macro-by-example notation augmented with a syntax specification \cite{culpepper10}
(colons \\denote syntax classes, which are first-class and can be built from the ground up).
\end{adjustwidth}
\end{listing}
\end{figure*}

\begin{figure*}
\hskip3.95cm
\normalsize Figure 1. Assorted implementations of the \small \texttt{let} \normalsize macro in Lisp dialects
% saves an empty line
% \begin{center}
% \normalsize Figure 1. Assorted implementations of the \texttt{let} macro in Lisp dialects
% \end{center}
\end{figure*}

\begin{figure*}[t]
\begin{listing}
\normalsize

\begin{tabular}{p{5.4cm} p{15cm}}\\
 &
\begin{verbatim}
(defmacro or (x y)
  '(let ((temp ,x))
    (if temp temp ,y)))

(or 42 "the result is 42")
\end{verbatim}
\end{tabular}

\begin{center}
a) A simple yet erroneous implementation of the \texttt{or} macro
\end{center}

\begin{tabular}{p{8.5cm} p{8.5cm}}\\
\begin{verbatim}
(let ((temp "451 Fahrenheit"))
  (or null temp))

(let ((temp "451 Fahrenheit"))
  (let ((temp null))
    (if temp temp temp)))
\end{verbatim}
&
\begin{verbatim}
(let ((if hijacked))
  (or true false))

(let ((if hijacked))
  (let ((temp true))
    (if temp temp false)))
\end{verbatim}\\
b) Violation of hygiene \cite{kohlbecker86}: binding established during expansion
affects call site
&
c) Violation of referential transparency \cite{dybvig92}: binding established during expansion
is affected by call site
\end{tabular}

\begin{tabular}{p{2.8cm} p{15cm}}\\
 &
\begin{verbatim}
(define-syntax forever
  (syntax-rules ()
    ((forever body ...)
     (call/cc (lambda (abort)
                (let loop () body ... (loop)))))))

(forever (print 4) (print 2) (abort))
\end{verbatim}
\end{tabular}

\begin{adjustwidth}{2.5cm}{0pt}
d) Intentional variable capture: infinite looping construct \texttt{forever} provides
a predefi-\\ned identifier to break from the loop. \texttt{abort} is introduced during
macro expansion,\\ but it should be visible to the body of the loop, which
comes from the macro call site.
\end{adjustwidth}

\end{listing}
\end{figure*}

\begin{figure*}
\hskip5.05cm
\normalsize Figure 2. Intentional and unintentional variable capture
% saves an empty line
% \begin{center}
% \normalsize Figure 2. Intentional and unintentional variable capture
% \end{center}
\end{figure*}

\section{Examples}

\small \texttt{let} \normalsize is a language construct typical to functional programming. It introduces
a scope for a computation and brings temporary variables with provided values into that scope.
To implement \small \texttt{let} \normalsize the compiler might wrap the computation in a lambda abstraction
and apply it right away (Figure 1a).

This notion cannot be abstracted procedurally, because the body of the computation typically
contains free variables. However textual abstraction fits the bill, because macros can manipulate
the program on a level where bindings don't exist and therefore don't impose restrictions.
To set up a stage for further chapters, let's implement the \small \texttt{let} \normalsize macro in Lisp.

The most straightforward solution to the problem is a low-level macro transformer (Figure 1b).
It takes an S-expression that represents a macro application, destructures it using standard
list manipulation functions, such as \small \texttt{car} and \small \texttt{cdr}\normalsize,
and creates a new S-expression with \small \texttt{cons}\normalsize.
Even in this simple example this notation is very noisy. It's quite difficult
to figure out expected shapes of input and output expressions from the imperative algorithm.

Quasiquotes \cite{bawden99} make it possible to reduce obscurity of the macro by providing
a domain-specific language for syntax templates (Figure 1c). The quasiquote operator (\texttt{`})
demarcates a static template. Quasiquoted code is inserted verbatim into the output
(that's why there's no longer need in explicit \small \texttt{cons}\normalsize'ing).
Unquote operators (\texttt{,}
and \texttt{,@}) interrupt a quasiquote, producing "holes" filled in with dynamically calculated
data. For example, for \small \texttt{let} \normalsize we statically know the shape of code to produce
(an application of a lambda abstraction) - this makes up the static part of the quasiquote. On the other hand,
body and parameters of the lambda as well as the arguments of the application may vary from
expansion to expansion - this is the dynamic part.

Another simplification of the macro can be achieved with MBE, the macro-by-example notation
\cite{kohlbecker87}. In their seminal work Kohlbecker and Wand came up with a specification
of a pattern matcher that matches singular and repetitive parts of S-expressions.
Identifiers which appear in the input pattern are treated as pattern variables,
ellipses (\texttt{...}) used as a last element of a list that contains pattern variables
denote repetition and, when nested, can capture lists of arbitrary depth.
The revised version of the \small \texttt{let} \normalsize macro is particularly minimalistic (Figure 1d).

A recent development of the MBE syntax has been proposed by Culpepper and Felleisen \cite{culpepper10}.
Their refinement addresses the need for principled input validation and error reporting. Indeed,
MBE covers the success path, but doesn't help with detecting errors. For example,
duplicate identifier names as in \small \texttt{(let ((x 40) (x 2)) (print (+ x y))} \normalsize will go
unnoticed until the compiler gets to the resulting lambda form, which will produce
confusing error messages. Authors enhance MBE with both declarative and procedural
means of validation (Figure 1e). Colons next to the names of pattern variables denote syntax classes,
which put restrictions on the shape of the variables, \small \texttt{\#:fail-when} \normalsize
clauses can contain validation code and error messages (this doesn't cover all the capabilities
of syntax specifications, please, refer to \cite{culpepper10} for details).
These validation facilities can be packed into custom syntax classes, which can be built from the ground up.

\section{Bindings}

Syntactic macros operate on ASTs, so they cannot produce syntactically invalid code,
but nothing prevents such macros from making semantic errors, i.e. expanding into
syntactically valid nonsense.
The most blatant mistakes will result in type errors, however there's an entire class
of oversights that might silently change the behavior of the program.

When writing procedures in lexically scoped languages, programmers typically don't need to think
about name clashes between variables declared inside procedures and at their call sites.
Except for recursion, scopes of procedure bodies and procedure call sites are different, so
neither variables defined inside the procedures can change the semantics at the call site, nor vice versa.

In a macro-enabled language, especially in presense of quasiquotes which make
macros look like normal procedures, this intuition becomes compromised,
because after macro expansion the scopes of macro definition and macro call site
are mechanically merged by the expander.

Conider the \small \texttt{or} \normalsize macro shown on Figure 2a. This macro picks one of its two arguments
based on the truthiness of the first argument. To prevent double evaluation, the macro introduces
a temporary variable \small \texttt{temp} \normalsize that holds the result of evaluation of the first argument.
The temporary variable is then tested with \small \texttt{if} \normalsize, which selects the resulting value.

As simple as this code can be, it is also incorrect, having two potential bugs.

The first problem happens when the call site defines its own variable named \small \texttt{temp} \normalsize
and relies on it in the second argument of a call to \small \texttt{or} \normalsize (Figure 2b).
After the expansion, two \small \texttt{temp}\normalsize s clash,
which produces incorrect results if the first argument of the call is falsy.
This problem is dubbed \emph{hygiene violation}, and the macro is said to introduce the temporary
variable \emph{unhygienically}.

In his thesis \cite{kohlbecker86} Kohlbecker defines the hygiene condition
and presents a macro system that automatically prevents such naming collisions:
\begin{quote}
Hygiene Condition for Macro Expansion.

Generated identifiers that become binding instances in the
completely expanded program must bind only identifiers that are
generated during the same transcription step.
\end{quote}

The second problem is in a sense dual to the first one. It happens when the call site
redefines one of the procedures or macros used in the expansion. For example, on Figure 2c
the call site hijacks the meaning of \small \texttt{if}\normalsize, which destroys the original intent of the
macro author. This is a \emph{referential opaqueness} problem.

Dybvig et al. \cite{dybvig92} build on Kohlbecker's work and devise a macro expander that
automatically avoids this class of errors:
\begin{quote}
Macros defined in [our] high-level specification language are
referentially transparent in the sense that a macro-introduced
identifier refers to the binding lexically visible where the macro
definition appears rather than to the top-level binding or to the
binding visible where the macro call appears.
\end{quote}

This notion is also called \emph{cross-stage persistence} in a sense that bindings in place
at the compilation stage are persisted into the runtime stage. When viewed in this light,
it becomes apparent that referential transparency for macros can only work for references
to top-level definitions. Therefore common practice is to report cross-stage references to local variables
as errors \cite{dybvig92}.

Despite the conveniency of automatic segregation of the scopes of macro definitions and macro call sites,
at times it is necessary to have them melded.
A looping macro \small \texttt{forever} \normalsize from Figure 2d demonstrates the need for this.
Expansion of this macro is an infinite loop that provides a magic identifier \small \texttt{abort} \normalsize
to exit the loop. In this case automatic hygiene facility will do harm, making the body of the loop
unable to see the loop control lever.

Because of similar situations some programmers argue in favor of manual control over
the scoping discipline, proposing manual (or macro-powered!) renaming of identifiers whose capture would be
undesired \cite{clinger91,hoyte08}.

Another popular line of thought favors further empowerment of macro systems
to minimize the necessity for low-level tinkering.
In fact, one of the macro systems reviewed below provides a design pattern \cite{barzilay11}
that solves the \small \texttt{abort} \normalsize challenge in a fully hygienic way (i.e. without introducing
identifiers with manually provided names).

\begin{figure*}[t]
\begin{listing}
\normalsize

\begin{tabular}{p{2.0cm} p{15cm}}\\
 &
\begin{verbatim}
-| fun pow n = <fn x =>
    ~(if n = 0 then <1> else <x * (~(pow (n - 1)) x)>)>;
val pow = fn : int -> <int -> int>

-| val cube = (pow 3);
val cube = <(fn a => x %* x %* x %* 1)> : <int -> int>

-| (run cube) 5;
val it = 125 : int
\end{verbatim}
\end{tabular}

\begin{adjustwidth}{2.5cm}{0pt}
a) Strict typechecking \cite{taha99}: each quasiquote is typechecked individually and is assigned\\
a "code of something" type (\small \texttt{<a>} \normalsize stands for code that evaluates to a
value of type \small \texttt{a})\normalsize.\\
\end{adjustwidth}

\begin{tabular}{p{2.0cm} p{15cm}}\\
 &
\begin{verbatim}
[| 'a' + True |] -- rejected

printf :: String -> Expr -- allowed
$(printf "Error: %s on line %d") "urk" 341

f :: Q Type -> Q [Dec] -- rejected
f t = [d| data T = MkT $t; g (MkT x) = g+1 |]
\end{verbatim}
\end{tabular}

\begin{adjustwidth}{2.5cm}{0pt}
b) Lenient typechecking \cite{sheard02}: quasiquotes are sanity checked to prevent blatant errors\\
and are required to have their bindings established before macro expansions kick in.\\
No additional checks are done, and all quasiquotes get the same type-agnostic
\small \texttt{Expr} \normalsize type.
\end{adjustwidth}

\begin{tabular}{p{2.0cm} p{15cm}}\\
 &
\begin{verbatim}
macro using(name, expr, body) {
  <[
    def $name = $expr;
    try { $body } finally { $name.Dispose() }
  ]>
}
using(db, Database("localhost"), db.LoadData())
\end{verbatim}
\end{tabular}

\begin{adjustwidth}{2.5cm}{0pt}
c) No typechecking \cite{skalski05, skalski04}: quasiquotes are not typechecked at all until a
macro\\ that uses them expands, after which the result is typechecked as a whole. This provides\\
maximum freedom, permitting even unquotes in binding positions
(like in \small \texttt{def \$name}\normalsize).
\end{adjustwidth}

\end{listing}
\end{figure*}

\begin{figure*}
\hskip5.48cm
\normalsize Figure 3. Typechecking strategies for quasiquotes
% saves an empty line
% \begin{center}
% \normalsize Figure 3. Typechecking strategies for quasiquotes
% \end{center}
\end{figure*}

\section{Typechecking}

An important feature of macros is that macro expansion happens before the code is executed (below in
the paper this phase is referred to as \emph{compile-time}, despite that, strictly speaking, interpreted
languages can also support macros). Therefore macro expansions can be checked for
absense of semantic errors (to the extent supported by the language) in an automatic fashion,
without requiring effort from the programmer.

What's even better is that there is another line of defense.
Macros operate on syntax objects, which are snippets of the code to be,
thus it might be a good idea to typecheck the code represented by these snippets
in advance. This amounts to extending typechecking to the before-compile-time phase
(as in compile-time of macros themselves).

One approach is to treat quasiquotes similarly to functions, requiring all free variables
to be bound in the enclosing lexical scope and assigning quasiquotes a definitive type (Figure 3a).
This strategy is used in MetaML \cite{taha99}, a statically typechecked
language with special support for program generation.

All quasiquotes in MetaML are typechecked individually
and receive one of the \small \texttt{<a>} \normalsize
types (which means "code that evaluates to a value of type \small \texttt{a}\normalsize").
For example, on Figure 3a the quasiquote in the body of \small \texttt{pow} \normalsize
represents a function from \small \texttt{int} \normalsize to \small \texttt{int}\normalsize, so it has the
\small \texttt{<int -> int>} \normalsize type.

Programs in MetaML can generate and run programs at runtime, these programs can do the same, and so on.
Quasiquotes here represent templates of the programs to be generated and delineate execution stages.
Due to its specific nature, MetaML requires strong guarantees from the type system.
At runtime, if program generation fails because of an error in the meta-program,
it's too late to report typechecking errors.

Unfortunately this means that MetaML has to give up the freedom of arbitrary manipulations
of syntax objects, the most important of which are destructuring and introduction of new bindings.
(e.g. in MetaML it’s impossible to express \small \texttt{let} \normalsize as a macro).

MacroML \cite{ganz01,taha03}, a macro system built atop of MetaML,
provides a clever way to partially alleviate this restriction, letting
the programmer introduce new binding constructs in a very controlled setting.
This is achieved by having the type system track
not only the regular environments containing declared macros and variables, but also
a so called body parameter environment, which remembers which parts of quasiquotes
see which variables introduced by those quasiquotes.

This still doesn't allow arbitrary manipulations, e.g. similarly to MetaML there is no way
to pull apart a syntax object and reassemble it using a quasiquote as a template
(as done in the \small \texttt{let} \normalsize macro on Figure 1).

Therefore later research on macros focused on relaxing the typechecking discipline of MetaML.

Template Haskell \cite{sheard02} employs a lenient typechecking strategy for meta-programs
(but not for expanded programs, which are typechecked with the usual rigor of Haskell).

Unlike in MetaML, the type of syntax objects is non-polymorphic - \small \texttt{Expr} \normalsize,
not \small \texttt{Expr a} \normalsize - and Template Haskell doesn't try to provide up-front
guarantees that meta-programs produce well-typed code. For example, well-typedness of the
\small \texttt{printf} \normalsize macro application (Figure 3b) is checked on the spot,
immediately after the expansion, not ahead of time.

However Template Haskell does require the bindings of variables used in quasiquotes to be resolved statically.
This is done to enforce the "static scope is absolute" design principle
(which is a rehash of hygiene and referential transparency conditions outlined in the previous chapter)
as stated in \cite{sheard03}:
\begin{quote}
If an occurrence of a variable \small \texttt{x} \normalsize looks as if it is lexically bound
to an enclosing binding for \small \texttt{x}\normalsize,
then it is so bound, and no Template-Haskell transformation can threaten that binding.
\end{quote}

Unfortunately this lightweight typecheck has proven to impose unnecessary limitations.
For example, the meta-program \small \texttt{f} \normalsize from Figure 3b
that uses another meta-program \small \texttt{t} \normalsize to insert a type declaration
in a quasiquote is rejected. This happens because the compiler cannot prove that
the usage of this type declaration (\small \texttt{MkT x}\normalsize)
is correct for all possible \small \texttt{t}\normalsize s.

It is due to this reason that Nemerle \cite{skalski05, skalski04},
a macro-enabled language inspired by Template Haskell,
has ceased typechecking quasiquotes altogether. As we will see below this doesn't prevent
the macro expander in Nemerle from being hygienic, but with respect to typechecking
this brings the evolution to square one.

\section{Case study}

This chapter discusses Template Haskell \cite{sheard02},
Nemerle \cite{skalski04,skalski05} and Racket \cite{barzilay11,flatt10}
in the light of hygiene and referential transparency.

Syntax objects in all three presented macro systems can be manipulated
both at low level (using raw construction of corresponding data structure)
and at high level (with the help of quasiquotation facilities).

However in this survey we will only be studying and evaluating high-level notations and APIs.
Low-level facilities are undoubtedly useful for getting things done,
but our intent here is to get to the bottom of abstractions
introduced by the scrutinees.

We will also take for granted the ability of meta-programs to ask the expander
for the information about the parts of the program that lie outside of
a macro application current to a given expansion (the \small \texttt{reify} \normalsize family of functions
in Template Haskell, \small \texttt{Macros.ImplicitCTX} \normalsize in Nemerle and the
\small \texttt{syntax-local} \normalsize family of functions in Racket) or
for metadata of syntax objects (e.g. their types, scopes, etc).

Another aspect that we will omit from consideration is the way the macro expander
acquires and loads binary code of macros. For the record, all three reviewed languages are compiled,
but they differ in their treatment of the situation. Template Haskell and Nemerle require separate
compilation, whereas Racket allows joint compilation with extra precautions against cross-stage
pollution of top-level scope \cite{flatt02}.

Finally, in Nemerle and Racket macros can change the syntax of the language.
As mentioned in the introduction, syntax extensibility is a lively
research topic, but it is outside the scope of this study.

\subsection{Template Haskell}

Template Haskell is a macro-enabled extension to a statically typed functional programming language.
% The main contributions of this macro system are hygienic translation of quasiquotes with the use of
% a monad that encapsulates generation of fresh names and a relaxed typechecking algorithm for
% meta-programs and macro expansions.
Macro transformers in Template Haskell are represented by regular functions that produce
data of type \small \texttt{Expr}\normalsize. Macro expansion is triggered explicitly by
unquoting outside quasiquotes - either with \small \texttt{\$} \normalsize which unquotes expressions,
or with \small \texttt{splice} \normalsize which unquotes declarations.

\small
\begin{verbatim}
printf :: String -> Expr
printf s = gen (parse s) [| "" |]

gen :: [Format] -> Expr -> Expr
gen [] x = x
gen (D   : xs) x =
  [| \n-> $(gen xs [| $x++show n |]) |]
gen (S   : xs) x =
  [| \s-> $(gen xs [| $x++s |]) |]
gen (L s : xs) x =
  gen xs [| $x ++ $(lift s) |]

$(printf "Error: %s on line %d") msg line
\end{verbatim}
\normalsize

The snippet above illustrates the introduced concepts. \small \texttt{printf} \normalsize and
\small \texttt{gen} \normalsize are functions that operate on syntax objects. In particular,
\small \texttt{printf} \normalsize generates a lambda abstraction from a C-like string format
specification using helpers \small \texttt{parse} \normalsize (whose declaration is
not present in a snippet) and \small \texttt{gen}\normalsize.
Quasiquotes are encoded with \small \texttt{[|...|]} \normalsize brackets and can contain arbitrary
Haskell code (to be precise, the notation for quoted patterns, declarations and types is slightly different,
but it doesn't make much difference in the context of this discussion):

Quasiquotes in Template Haskell are hygienic and referentially transparent. The former is achieved
by alpha-renaming bindings introduced in quasiquotes, while the latter is warranted by keeping track
of original names that unambiguously refer to top-level variables, even if such variables are not visible
at macro use site. A curious fact is that Template Haskell also provides cross-stage persistence for
local variables that conform to the \small \texttt{Lift} \normalsize type class.

A very interesting aspect of Template Haskell is its implementation of hygiene through alpha-renaming,
an inherently stateful operation, in the pure Haskell environment.

\small \texttt{Expr}\normalsize,
the type of quasiquotes, is in fact a synonym to \small \texttt{Q Exp}\normalsize,
a type of low-level syntactic data structures lifted into the quotation monad \small \texttt{Q}\normalsize
that provides usual monadic operations along with \small \texttt{gensym :: String -> Q String}\normalsize,
a fresh name generator.
In this setting quasiquotes are mechanically translated into calls to low-level constructors wrapped in
\small \texttt{Q} \normalsize with binders renamed using \small \texttt{gensym} \normalsize and bindees
either following the renamed binders or persisted across stages as outlined above:

\small
\begin{verbatim}
cross2a :: Expr -> Expr -> Expr
cross2a f g = [| \(x,y) -> ($f x, $g y) |]

cross2c :: Expr -> Expr -> Expr
cross2c f g =
do { x <- gensym "x"
   ; y <- gensym "y"
   ; ft <- f
   ; gt <- g
   ; return (Lam [Ptup [Pvar x, Pvar y]]
                 (Tup  [App ft (Var x)
                       ,App gt (Var y)]))
}
\end{verbatim}
\normalsize

This approach is ingenious, but unfortunately it has limitations.

First of all, in general case roles
of identifiers in quasiquotes (binder, bound within quasiquote, bound outside) cannot be determined
before macro expansion. For example, in \small \texttt{[|~f~\$p~=~x~+~y~|]}\normalsize,
variables \small \texttt{x} \normalsize and \small \texttt{y} \normalsize are on the fence
until the quasiquote is evaluated. To the contrast, the translation algorithm requires an up front decision,
which makes splices into binding positions illegal.

Secondly, there is no way to opt out of hygiene without dropping to low level.
Even worse, since the translation algorithm effectively forbids unbound free variables in quasiquotes,
every call site of a macro that introduces variables visible outside would have to drop to low level.
For instance, users of the \small \texttt{forever} \normalsize macro (Figure 2d) won't be able to write
\small \texttt{\$(forever [| abort |])}\normalsize, but will be forced to express this as
\small \texttt{\$(forever [| \$(var "abort") |])}\normalsize.

\section{Nemerle}

Nemerle is a general purpose object-oriented/functional programming language.
Macros in Nemerle are introduced with the keyword \small \texttt{macro} \normalsize
and can use quasiquotes (\small \texttt{<[...]>}\normalsize) to pattern match
and compose syntax objects. Expansions are triggered automatically when
the compiler encounters a function application, and the callee is deemed to be a macro.
Nemerle also supports macro attributes that expand when put on declarations.
Such macros can modify the inheritance chain of classes, generate new members, etc.

The macro \small \texttt{<->} \normalsize shown below swaps values of two expressions
like in \small \texttt{x <-> arr[2]}\normalsize. In the snippet we can see how quasiquotes
can be used to pattern match against syntax objects, extract their parts and reassemble
producing new syntax objects. The \small \texttt{[Hygienic]} \normalsize attribute is discussed later.

\small
\begin{verbatim}
[Hygienic]
def cache(e: Expr): Expr * Expr
{
  | <[ $obj.$mem ]> =>
    (<[ def tmp = $obj ]>, <[ tmp.$mem ]>)
  | <[ $tab[$idx] ]> =>
    (<[ def (tmp1, tmp2) = ($tab, $idx) ]>,
     <[ tmp1[tmp2] ]>)
  | _ => (<[ () ]>, e)
}

macro @<->(e1, e2) {
  def (cached1, safe1) = cache(e1);
  def (cached2, safe2) = cache(e2);
  <[
    $cached1;
    $cached2;
    def tmp = $safe1;
    $safe1 = $safe2;
    $safe2 = tmp;
  ]>
}
\end{verbatim}
\normalsize

% Note that unlike Template Haskell and like Lisp, Nemerle
% doesn't require explicit nudges to trigger macro expansion
% and automatically expands function applications if the callee is deemed to be a macro.
% This simplifies macro calls at a cost of having macros as second-class citizens
% (macros cannot be passed around, because all references to a macro are immediately expanded).
% In any case, macros can call arbitrary functions, so it's only a facade (i.e. macro definitions)
% that is restricted. The internals (i.e. macro implementations) can use all the abstraction mechanisms
% provided by the language, so second-class nature of macros is usually considered to be a minor inconvenience.

From the example above it is apparent that quasiquotes in Nemerle
can have unbound free variables and even allow splicing into binding positions.
Nevertheless Nemerle is hygienic and referentially transparent (except for cross-stage
references to local variables, which are prohibited). This is achieved with
an algorithm of impressive simplicity and power.

To separate lexical scopes of macro definitions and macro call sites, Nemerle colors the
identifiers in the program. By default each macro expansion gets a unique color, different from
the color of normal code, that gets assigned to the symbols it introduces. After the program
is fully expanded Nemerle resolves bindings, i.e. for every bindee the compiler looks for
a nearest preceding binder with the same color, performing alpha-renaming as necessary.
If there's no such binder, the name is looked up
in a global environment that corresponds to the definition site of the bindee (such environments
are enclosed in every symbol).

An important characteristic of this algorithm is that colors are fully customizable.
Using an API exposed by the expander it is possible to recolor individual symbols and
entire bunches into arbitrary colors. It is also possible to share colors between macros
to ensure that selected parts of corresponding expansions can see each other.
Finally Nemerle supports polychromatic symbols that bind
to the nearest definitions with the same name regardless of the color.

Nemerle also provides most frequently used patterns of tweaks to bindings as the part of the core language.
\small \texttt{\$(name:~usesite)}\normalsize introduces a symbol colored identically to the macro call site.
\small \texttt{[Hygienic]} \normalsize attribute on a function (usually used on helper functions that can be
called multiple times from the same macro) colors binders that come from that function is a color that
is generated afresh every invocation.

\section{Racket}

Racket is a dynamically typed descendant of Scheme. Being a product of several decades of evolution,
Racket uses Scheme's \small \texttt{syntax-case} \normalsize \cite{dybvig92}, a hygienic and referentially
transparent macro system, augmented with syntax classes \cite{culpepper10}.

To prevent inadvertent captures \small \texttt{syntax-case} \normalsize uses Dybvig's algorithm,
which is similar to Nemerle's color algorithm outlined above (in fact, the causal relation is inversed -
the algorithm in Nemerle is inspired by Dybvig's work). Racket's algorithm generates fresh marks for
every macro expansion, assigns these symbols introduced by expansions and then uses the marks to
resolve bindings and perform alpha-renamings if necessary.

Much like in Nemerle, affiliation of symbols can be customized. The \small \texttt{datum->syntax} \normalsize
function can be used to put an arbitrary S-expression into a lexical context of an arbitrary syntax object,
including that of macro use site and macro definition site.

The snippet below creates a non-hygienic symbol \small \texttt{abort} \normalsize in the lexical context
of the macro use site (\small \texttt{\#'forever}\normalsize) and then introduces it in macro expansion
as a binder, making it possible for a macro argument to see an identifier in the generated code:

\small
\begin{verbatim}
(define-syntax (forever stx)
  (syntax-case stx ()
    ((forever body ...)
     (with-syntax
      ((abort (datum->syntax #'forever 'abort)))
       #'(call/cc (lambda (abort)
           (let loop () body ... (loop))))))))

(forever (print 4) (print 2) (abort))
\end{verbatim}
\normalsize

\begin{thebibliography}{99}

\bibitem{sheard02}
T.~Sheard and S.~Peyton Jones,
Template meta-programming for Haskell.
ACM SIGPLAN Notices, 2002.

\bibitem{skalski04}
K.~Skalski, M.~Moskal and P.~Olszta,
Meta-programming in Nemerle.
Generative Programming and Component Engineering, 2004.

\bibitem{barzilay11}
E.~Barzilay, R.~Culpepper and M.~Flatt,
Keeping it Clean with Syntax Parameters.
Scheme and Functional Programming Workshop, 2011.

\bibitem{odersky10}
M.~Odersky, L.~Spoon and B.~Venners,
Programming in Scala 2nd Edition.
Artima, 2010.

\bibitem{kohlbecker86}
E.~Kohlbecker,
Syntactic Extensions in the Programming Language Lisp.
PhD thesis, Indiana University, 1986.

\bibitem{sheard01}
T.~Sheard,
Accomplishments and Research Challenges in Meta-Programming.
Semantics, Applications, and Implementation of Program Generation, 2001.

\bibitem{atkinson11}
K.~Atkinson and M.~Flatt,
Adapting Scheme-like macros to a C-like language.
Scheme and Functional Programming, 2011.

\bibitem{allen09}
E.~Allen, R.~Culpepper, J-D.~Nielsen, J.~Rafkind and S.~Ryu,
Growing a Syntax.
Foundations of Object-Oriented Languages, 2009.

\bibitem{erdweg11}
S.~Erdweg, T.~Rendel, C.~K\"{a}stner and K.~Ostermann,
SugarJ: Library-based syntactic language extensibility,
Object-Oriented Programming, Systems, Languages, and Applications, 2011.

\bibitem{dybvig92}
R.~Dybvig, R.~Hieb and  C.~Bruggeman,
Syntactic Abstraction in Scheme.
Lisp and Symbolic Computations, 1992.

\bibitem{bawden99}
A.~Bawden,
Quasiquotation in Lisp.
Partial Evaluation and Program Manipulation, 1999.

\bibitem{kohlbecker87}
E.~Kohlbecker, M.~Wand,
Macro-by-Example: Deriving Syntactic Transformations from their Specifications.
Principles of Programming Languages, 1987.

\bibitem{culpepper10}
R.~Culpepper, M.~Felleisen,
Fortifying Macros.
International Conference on Functional Programming, 2010.

\bibitem{hoyte08}
D.~Hoyte,
Let Over Lambda.
Lulu, 2008.

\bibitem{clinger91}
W.~Clinger,
Hygienic macros through explicit renaming.
ACM SIGPLAN Lisp Pointers, 1991.

\bibitem{taha99}
W.~Taha,
Multi-Stage Programming: Its Theory and Applications.
PhD Thesis, Oregon Graduate Institute, 1999.

\bibitem{ganz01}
S.~Ganz, A.~Sabry, W.~Taha,
Macros as Multi-Stage Computations: Type-Safe, Generative, Binding Macros in MacroML.
International Conference on Functional Programming, 2001.

\bibitem{taha03}
W.~Taha and P.~Johann,
Staged Notational Definitions.
Generative Programming and Component Engineering, 2003.

\bibitem{skalski05}
K.~Skalski,
Syntax-Extending and Type-Reflecting Macros in an Object-Oriented Language.
MsC Thesis, University of Wroc{\l}aw, 2005.

\bibitem{sheard03}
T.~Sheard and S.~Peyton Jones,
Notes on Template Haskell Version 2.
Glasgow Haskell Compiler, 2003.

\bibitem{flatt10}
M.~Flatt and PLT,
Reference: Racket,
PLT Inc., 2010.

\bibitem{flatt02}
M.~Flatt,
Composable and Compilable Macros: You Want it When?
International Conference on Functional Programming, 2002.

% \bibitem{bawden00}
% A.~Bawden,
% First-class Macros Have Types.
% Principles of Programming Languages, 2000.

\end{thebibliography}

\end{document}
