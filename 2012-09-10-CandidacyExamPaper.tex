\documentclass[10pt,journal,a4paper]{IEEEtran}

\newcommand\MYhyperrefoptions{bookmarks=true,bookmarksnumbered=true,
pdfpagemode={UseOutlines},plainpages=false,pdfpagelabels=true,
colorlinks=true,linkcolor={black},citecolor={black},pagecolor={black},
urlcolor={black},
pdftitle={Metaprogramming with Macros},
pdfsubject={Metaprogramming with Macros},
pdfauthor={Eugene Burmako},
pdfkeywords={metaprogramming, macros, quasiquotes, hygiene, referential transparency}}

\begin{document}

\title{Metaprogramming with Macros}

\author{Eugene Burmako% <-this % stops a space
\\ \hskip90pt LAMP, I\&C,  EPFL % no idea why I need an hskip, but I'm not a latex whiz
\thanks{\normalsize Proposal submitted to committee: September 3rd, 2012; Candidacy exam date: September 10th, 2012; Candidacy exam committee: Christoph Koch, Martin Odersky, Viktor Kuncak.}%
\IEEEcompsocitemizethanks{\IEEEcompsocthanksitem }
\thanks{\large This research plan has been approved:}%
\IEEEcompsocitemizethanks{\IEEEcompsocthanksitem \large}
\IEEEcompsocitemizethanks{\IEEEcompsocthanksitem \large}
\IEEEcompsocitemizethanks{\IEEEcompsocthanksitem \large Date:\hfill------------------------------------}
\IEEEcompsocitemizethanks{\IEEEcompsocthanksitem \large}
\IEEEcompsocitemizethanks{\IEEEcompsocthanksitem \large}
\IEEEcompsocitemizethanks{\IEEEcompsocthanksitem \large Doctoral candidate:\hfill------------------------------------}
\IEEEcompsocitemizethanks{\IEEEcompsocthanksitem \footnotesize \hfill                      (name and signature)\hspace{1.5cm}\hfill}
\IEEEcompsocitemizethanks{\IEEEcompsocthanksitem \large}
\IEEEcompsocitemizethanks{\IEEEcompsocthanksitem \large}
\IEEEcompsocitemizethanks{\IEEEcompsocthanksitem \large Thesis director:\hfill------------------------------------}
\IEEEcompsocitemizethanks{\IEEEcompsocthanksitem \footnotesize \hfill                      (name and signature)\hspace{1.5cm}\hfill}
\IEEEcompsocitemizethanks{\IEEEcompsocthanksitem \large}
\IEEEcompsocitemizethanks{\IEEEcompsocthanksitem \large}
\IEEEcompsocitemizethanks{\IEEEcompsocthanksitem \large Thesis co-director:\hfill------------------------------------}
\IEEEcompsocitemizethanks{\IEEEcompsocthanksitem \footnotesize (if applicable)\hfill  (name and signature)\hspace{1.5cm}\hfill}
\IEEEcompsocitemizethanks{\IEEEcompsocthanksitem \large}
\IEEEcompsocitemizethanks{\IEEEcompsocthanksitem \large}
\IEEEcompsocitemizethanks{\IEEEcompsocthanksitem \large Doct. prog. director:\hfill------------------------------------}
\IEEEcompsocitemizethanks{\IEEEcompsocthanksitem \footnotesize (R. Urbanke)  \hfill                    (signature)\hspace{1.5cm}\hfill}
\IEEEcompsocitemizethanks{\IEEEcompsocthanksitem \large}
\IEEEcompsocitemizethanks{\IEEEcompsocthanksitem \large}
\IEEEcompsocitemizethanks{\IEEEcompsocthanksitem \tiny EDIC-ru/05.05.2009}
}

\markboth{EDIC Research Proposal}%
{Shell \MakeLowercase{\textit{et al.}}: EDIC Research Proposal}

\IEEEcompsoctitleabstractindextext{%
\begin{abstract}
Macros realize the notion of textual abstraction.
Textual abstraction consists of recognizing pieces of text
that match a specification and replacing them according
to a procedure.

In the focus of the study are syntactic abstractions in lexically
scoped programming languages. We identify the problems
of \emph{hygiene} and \emph{referential transparency}
and describe the solutions employed in Template Haskell \cite{sheard02},
Nemerle \cite{skalski04} and Racket \cite{barzilay11}.

We discuss integration of hygienic macros into statically typed languages
and propose to improve upon state of the art by providing
a type system for syntax templates and uncovering synergies with
high-level language features such as path-dependent types and implicits \cite{odersky10}.
\end{abstract}

\begin{IEEEkeywords}
metaprogramming, macros, quasiquotes, hygiene, referential transparency
\end{IEEEkeywords}}

\maketitle
\IEEEdisplaynotcompsoctitleabstractindextext
\IEEEpeerreviewmaketitle

\section{Introduction}

\IEEEPARstart{T}{his} template is not trying to suggest a specific
style for the content of your candidacy exam/thesis proposal paper; it
only concerns the style of the writeup itself.  This template is in LaTeX.
We encourage you to use it for your write-up.

%% the freaking quote about the notation from Kohlbecker's thesis page 22

\begin{thebibliography}{1}

\bibitem{sheard02}
T.~Sheard and S.~Peyton Jones,
Template meta-programming for Haskell.
ACM SIGPLAN Notices, 2002.

\bibitem{skalski04}
K.~Skalski, M.~Moskal and P.~Olszta,
Meta-programming in Nemerle.
Third International Conference on
Generative Programming and Component Engineering, 2004.

\bibitem{barzilay11}
E.~Barzilay, R.~Culpepper and M.~Flatt,
Keeping it Clean with Syntax Parameters.
Scheme and Functional Programming Workshop, 2011.

\bibitem{odersky10}
M.~Odersky, L.~Spoon and B.~Venners,
Programming in Scala 2nd Edition.
Artima, 2010.

\end{thebibliography}

\end{document}
